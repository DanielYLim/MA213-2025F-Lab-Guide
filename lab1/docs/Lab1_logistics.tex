%%%%%%%%%%%%%%%%%%%%%%%%%%%%%%%%%%%%
% Slide options
%%%%%%%%%%%%%%%%%%%%%%%%%%%%%%%%%%%%

% Option 1: Slides with solutions

\documentclass[slidestop,compress,mathserif]{beamer}
\newcommand{\soln}[1]{\textit{#1}}
\newcommand{\solnGr}[1]{#1}

% Option 2: Handouts without solutions

%\documentclass[11pt,containsverbatim,handout]{beamer}
%\usepackage{pgfpages}
%\pgfpagesuselayout{4 on 1}[letterpaper,landscape,border shrink=5mm]
%\newcommand{\soln}[1]{ }
%\newcommand{\solnGr}{ }

%%%%%%%%%%%%%%%%%%%%%%%%%%%%%%%%%%%%
% Style
%%%%%%%%%%%%%%%%%%%%%%%%%%%%%%%%%%%%

\input{../../lec_style.tex}
 

%%%%%%%%%%%%%%%%%%%%%%%%%%%%%%%%%%%%
% Preamble
%%%%%%%%%%%%%%%%%%%%%%%%%%%%%%%%%%%%

\title[Lecture 1]{MA213: Lab 1 Logistics}
\subtitle{Module 1: Exploratory Data Analysis and Study Design}
\author{OpenIntro Statistics, 4th Edition}
\institute{$\:$ \\ {\footnotesize Based on slides developed by Mine \c{C}etinkaya-Rundel of OpenIntro. \\
The slides may be copied, edited, and/or shared via the \webLink{http://creativecommons.org/licenses/by-sa/3.0/us/}{CC BY-SA license.} \\
Some images may be included under fair use guidelines (educational purposes).}}
\date{}

%%%%%%%%%%%%%%%%%%%%%%%%%%%%%%%%%%%%
% Begin document
%%%%%%%%%%%%%%%%%%%%%%%%%%%%%%%%%%%%

\begin{document}


%%%%%%%%%%%%%%%%%%%%%%%%%%%%%%%%%%%%
% Title page
%%%%%%%%%%%%%%%%%%%%%%%%%%%%%%%%%%%%

{
\addtocounter{framenumber}{-1} 
{\removepagenumbers 
\usebackgroundtemplate{\includegraphics[width=\paperwidth]{../../OpenIntro_Grid_4_3-01.jpg}}
\begin{frame}

\hfill \includegraphics[width=20mm]{../../oiLogo_highres}

\titlepage

\end{frame}
}
}

%%%%%%%%%%%%%%%%%%%%%%%%%%%%%%%%%%%%
% Sections
%%%%%%%%%%%%%%%%%%%%%%%%%%%%%%%%%%%%

%%%%%%%%%%%%%%%%%%%%%%%%%%%%%%%%%%%%
% \section{Lab Introduction}
%%%%%%%%%%%%%%%%%%%%%%%%%%%%%%%%%%%%

%%%%%%%%%%%%%%%%%%%%%%%%%%%%%%%%%%%

%%%%%%%%%%%%%%%%%%%%%%%%%%%%%%%%%%%
% Logistics
\begin{frame}
	\frametitle{Lab Time and Location}
	\begin{itemize}
		\item \hl{C1 Lab}:  
		\begin{itemize}
			\item Friday 9:05 AM - 9:55 AM : CA 213
		\end{itemize}
		\item \hl{C2 Lab}:  
		\begin{itemize}
			\item Friday 10:10 AM - 11:00 AM : COM 215
		\end{itemize}
		\item \hl{C3 Lab}:  
		\begin{itemize}
			\item Friday 12:20 PM - 1:10 PM : CDS 264
		\end{itemize}
		\item \hl{C4 Lab}:  
		\begin{itemize}
			\item Friday 1:25 PM - 2:15 PM : COM 215
		\end{itemize}
	\end{itemize}
\end{frame}

%%%%%%%%%%%%%%%%%%%%%%%%%%%%%%%%%%%
% People
\begin{frame}
	\frametitle{People}
	\begin{itemize}
		\item \hl{Instructor:} Prof. Emily Stephen
		\item \hl{Labs:} Dr. Yongho Lim
		\item \hl{Graduate Teaching Fellows (TFs):} 
		\begin{itemize}
			\item Matt Broe(Discussions)
			\item James Zheng Yang (Discussions)
			\item Matt Ludwig (Labs)
		\end{itemize}
		\item \hl{Undergraduate Learning Assistants (LAs):}
		\begin{itemize}
			\item Yao Lu (Labs, Lecture)
			\item Jack Hincks (Labs, Lecture)
		\end{itemize}
	\end{itemize}

	See the course website for office hours and contact info
\end{frame}

%%%%%%%%%%%%%%%%%%%%%%%%%%%%%%%%%%%
% Course Webpage

\begin{frame}
	\frametitle{Course Webpage} 
	\begin{itemize}
		\item \hl{Course website (Blackboard):} \url{https://learn.bu.edu} \\
		(Log in with your BU credentials)
		\item \hl{What's there:}
		\begin{itemize}
			\item Announcements
			\item Course documents: \textbf{Syllabus} (with Office Hours and Calendar), GenAI Policy, Learning Objectives
			\item Links to Textbook, Edfinity, Gradescope
			\item Course Forum
			\item Lecture slides and videos
			\item Lab materials
			\item Your gradebook
			\item Gradescope for post-lab activity questions
		\end{itemize}
	\end{itemize}	
\end{frame}




%%%%%%%%%%%%%%%%%%%%%%%%%%%%%%%%%%%
% Labs

\begin{frame}
	\frametitle{Structure of Labs} 
	Led by Yongho Lim, with support from TFs and LAs
	\begin{itemize}
		\item \hl{Weekly labs} on Fridays, starting this week
		\item Practice analyzing data and running simulations in R
		\item \hl{Skills labs}
		\begin{itemize}
			\item Work in pairs
			\item Analyze data and run simulations in R
			\item Post-lab questions to be submitted on Gradescope
		\end{itemize}
		\item \hl{Lab projects}
		\begin{itemize}
			\item Work in groups to explore real data
			\item Project 1: Data analysis video presentation
			\item Project 2: Statistical report
		\end{itemize}
	\end{itemize}
\end{frame}

%%%%%%%%%%%%%%%%%%%%%%%%%%%%%%%%%%%
% Assignments and Grading Structure

\begin{frame}
	\frametitle{Assignments in Lab}
	\begin{itemize}
		\item 7 \hl{Skills Labs}
			\begin{itemize}
				\item Post-lab activity (R file) to be submitted on Gradescope
				\item Due by 10 PM every Thursday, starting this week
				\item You can submit as many times as you want before the deadline
				\item You can work with your friends, but each person must submit their own work
				\item You can use Generative AI tools for \textbf{specific uses} with \textbf{proper documentation} (see syllabus and GenAI policy document)
			\end{itemize}
		\item 2 \hl{Lab Projects}	
			\begin{itemize}
				\item Project 1: Data analysis video presentation
				\item Project 2: Statistical report
				\item Both projects will have deliverables (Outline, Progress Report with R file, Final Submission)
				\item Both projects will be group projects (groups of 4)
			\end{itemize}
	\end{itemize}
\end{frame}




\begin{frame}
	\frametitle{Grading Structure}
	\begin{table}[ht]
	\centering
	\small
	\begin{tabular}{|l|c|c|c|}
	\hline
	 & \textbf{A} & \textbf{B} & \textbf{C} \\
	\hline
	Homeworks & 12/12 complete & 11/12 complete & 10/12 complete \\
	\hline
	Quizzes (Core) & 18/19 passed & 15/19 passed & 12/19 passed \\
	\hline
	Quizzes (Aux) & 9/11 passed & 6/11 passed & 0/11 passed \\
	\hline
	Skills Labs & 7/7 passed & Labs 1-6 passed & Labs 1-6 passed \\
	\hline
	Lab Projects & 2 satisfactory & 2 satisfactory & 2 satisfactory \\
	\hline
	Lectures & $>$34 attended & $>$28 attended & $>$20 attended \\
	\hline
	\end{tabular}
	\end{table}

	\hl{Additional factors:}
	\begin{itemize}
		\item Grades between letter grades will be determined by how close you are to the next letter grade
		\item Each ``unsatisfactory'' project will drop your course grade by a third of a letter grade (e.g. B becomes B-)
	\end{itemize}
\end{frame}

%%%%%%%%%%%%%%%%%%%%%%%%%%%%%%%%%%%

\begin{frame}
	\frametitle{Class Policies} 
	\begin{itemize}
		\item \hl{Attendance and participation:} 
		\begin{itemize}
			\item Expected to attend all lectures, labs, and discussions. 
			\item In-class activities and participation will be part of your grade.
			\item Email the instructor or your TF if you need to miss class/discussion/lab
		\end{itemize}
		\item \hl{Academic integrity:} 
		\begin{itemize}
			\item All work must be your own. 
			\item Collaboration is allowed on homeworks and labs/projects, but \textbf{not on quizzes}. 
			\item See syllabus for details.
		\end{itemize}
		\item \hl{Use of Generative AI tools (e.g. ChatGPT):}
		\begin{itemize}
			\item Quizzes: \textbf{Not allowed}.
			\item Homeworks: Use at your own discretion
			\item Labs/Projects: Allowed for \textbf{specific uses} with \textbf{proper documentation}.
			\item See syllabus and GenAI policy document for details.
		\end{itemize}
	\end{itemize}
\end{frame}

%%%%%%%%%%%%%%%%%%%%%%%%%%%%%%%%%%%
\begin{frame}
	\frametitle{Gen AI Policies}
	\begin{itemize}
		\item \textbf{Acceptable Uses} include:
		\begin{itemize}	
			\item Loading data
			\item Cleaning, formatting and organizing data tables
			\item Generating figures with ggplot
			\item Understanding error messages
		\end{itemize}		
	\end{itemize} 
	\begin{itemize}
		\item \textbf{Proper Documentation} include:
		\begin{itemize}	
			\item AI-written code with comments
			\item Examples of prompts used to generate code
		\end{itemize}		
	\end{itemize}
\end{frame}



%%%%%%%%%%%%%%%%%%%%%%%%%%%%%%%%%%%

\begin{frame}[fragile]
	\frametitle{GenAI Good Examples}
	Question : Import ``mtcars.csv'' to the variable carsdata. \\
	Create a new variable subdata containing only the data for automatic transmission cars with weight over 3000 lbs. \\ \\
	
	\begin{itemize}
	\item Your Prompt:
	\begin{scriptsize}
	\begin{verbatim}
		Give me R code for printing all the rows with automatic transmission cars 
		(am = 0) and weight over 3000 lbs (wt>3) from the mtcars dataset.
	\end{verbatim}
	\end{scriptsize}
	\item GenAI Response:
	
	\begin{scriptsize}
		\begin{verbatim}
		Certainly! Here is the R code to print all rows from the mtcars dataset 
		where the transmission is automatic (am == 0) and the weight (wt) 
		is greater than 3:

		\end{verbatim}
	\end{scriptsize}
\end{itemize}
\end{frame}

%%%%%%%%%%%%%%%%%%%%%%%%%%%%%%%%%%%

%%%%%%%%%%%%%%%%%%%%%%%%%%%%%%%%%%%

\begin{frame}[fragile]
	\frametitle{GenAI Good Examples}	
		Code generated by GenAI for the above prompt:
	\begin{scriptsize}
		\begin{verbatim}
		# Load the mtcars dataset
		data(mtcars)

		# Filter rows with automatic transmission (am == 0) and weight > 3
		result <- mtcars[mtcars$am == 0 & mtcars$wt > 3, ]

		# Print the filtered rows
		print(result)

		\end{verbatim}
	\end{scriptsize}
\end{frame}

%%%%%%%%%%%%%%%%%%%%%%%%%%%%%%%%%%%


%%%%%%%%%%%%%%%%%%%%%%%%%%%%%%%%%%%

\begin{frame}[fragile]
	\frametitle{GenAI Bad Examples}
	Question : Import ``mtcars.csv'' to the variable carsdata. \\
	Create a new variable subdata containing only the data for automatic transmission cars with weight over 3000 lbs.
	
	Your Prompt:
	\begin{scriptsize}
	\begin{verbatim}
	Import mtcars.csv to the variable carsdata. 
	Create a new variable subdata containing only the data for automatic 
	transmission cars with weight over 3000 lbs.
	\end{verbatim}
	\end{scriptsize}
	GenAI Response:
	
	\begin{scriptsize}
		\begin{verbatim}
		# Import mtcars.csv into the variable carsdata
	carsdata <- read.csv("mtcars.csv")

	# Create subdata with automatic transmission cars (am == 0) 
	# and weight over 3000 lbs (wt > 3.0)
subdata <- subset(carsdata, am == 0 & wt > 3.0)

		\end{verbatim}
	\end{scriptsize}
\end{frame}

%%%%%%%%%%%%%%%%%%%%%%%%%%%%%%%%%%%
%%%%%%%%%%%%%%%%%%%%%%%%%%%%%%%%%%%%
% End document
%%%%%%%%%%%%%%%%%%%%%%%%%%%%%%%%%%%%

\end{document}